A Wireless Sensor Network (WSN) consists of numerous sensor nodes
spread over a wide area to gather information and transmit it to a
sink node, which then sends it to the end user for analysis. Such
networks have a wide range of applications in areas like health,
military, security, wildlife monitoring, etc. These sensors gather
huge amounts of data, not all of which is important to the end
user. Only a few sensors at any given point of time have valuable
information.  Identifying such sensors consistently can help us
increase the Value of Information (VoI) of a system. We know that
sensors generally have limited resources in terms of memory capacity,
power supply and communication bandwidth. Hence, it is important to
take energy-efficiency into consideration while implementing an
approach.

In order to address the above problems, we propose a centralized
network model which makes use of Information Entropy to determine the
theoretical upper bound on the VoI available in a network. We also
provide a probabilistic sensor selection algorithm to consistently
select the most informative sensor, enabling the model to increase the
amount of information gathered from a network. Simulation results show
that our approach gathers more information, especially at low ping
rates, when compared with several state-of-the-art models. To our
knowledge, this is the first implementation of a VoI-based approach
for a centralized WSN that is intended to efficiently maximize the
amount of information gathered from the system.