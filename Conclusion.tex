In this paper, we started off by proposing a model that is best
suitable for WSNs that are going to be built on an energy budget. We
stated several reasons why a centralized architecture with a
high-energy central node would be beneficial in comparison to a
distributed architecture. Next, we proposed a novel entropy-based VoI
approach for a centralized network to determine the theoretical
maximum VoI present within a network. We also present a
probability-based sensor selection algorithm to consistently select
informative sensors, enabling the model to increase the amount of
information gathered from a network. Moreover, by varying the sampling
rate, we have shown that our model can extract a high amount of
information even at very low sampling rates.

The performance of our approach was compared to two state-of-the-art
techniques (clustering for a centralized WSN and distributed
self-censoring) currently being employed to increase both the amount
of information being gathered and the energy efficiency. We developed
a simulator emulating all three models, and results clearly show that
our approach gathers more information per ping, implying that it is
more energy-efficient than the comparison models. Hence, we conclude
that our model provides an energy-efficient approach for increasing
the information gathered from a centralized network.